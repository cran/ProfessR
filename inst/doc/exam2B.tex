\item {
\setlength{\itemsep}{0cm}
\setlength{\parskip}{.2cm}
\begin{samepage}
\textbf{
An event similar to a tsunami that can occur in lakes due to avalanches, earthquakes, and other mechanisms is called a \makebox[1cm]{\Rivpt\hrulefill\Rivpt}.
}
\begin{enumerate}
\item { 	seiche }
\item { 	splash }
\item { 	slosh }
\item { 	tidal wave }
\item { 	flood }
\end{enumerate}
\end{samepage}
}
\item {
\setlength{\itemsep}{0cm}
\setlength{\parskip}{.2cm}
\begin{samepage}
\textbf{
Over the last 7000 years there has been a \makebox[1cm]{\Rivpt\hrulefill\Rivpt}.
}
\begin{enumerate}
\item {  lowering of global average temperature totaling 2\ensuremath{^\circ}C }
\item {  raising of global average temperature totaling 2\ensuremath{^\circ}C }
\item {  constant global temperature, on average  }
\item {  None of these are correct. }
\end{enumerate}
\end{samepage}
}
\item {
\setlength{\itemsep}{0cm}
\setlength{\parskip}{.2cm}
\begin{samepage}
\textbf{
The destructive powers of tsunami result mainly from their \makebox[1cm]{\Rivpt\hrulefill\Rivpt}.
}
\begin{enumerate}
\item { 	incredible height }
\item { 	unpredictablility }
\item { 	cold water }
\item { 	momentum and long wavelength }
\item { 	none of these 		 }
\end{enumerate}
\end{samepage}
}
\item {
\setlength{\itemsep}{0cm}
\setlength{\parskip}{.2cm}
\begin{samepage}
\textbf{
The Older Dryas and Younger Dryas stages represented \makebox[1cm]{\Rivpt\hrulefill\Rivpt}.
}
\begin{enumerate}
\item {  two cooling stages during warmup from the last major glacial advance }
\item {  two of the more abrupt warming stages during the last glaciation }
\item {  one warming stage and one cooling stage during the last glaciation }
\item {  periods of dry climate  }
\end{enumerate}
\end{samepage}
}
\item {
\setlength{\itemsep}{0cm}
\setlength{\parskip}{.2cm}
\begin{samepage}
\textbf{
A strong El Nino is characterized by \makebox[1cm]{\Rivpt\hrulefill\Rivpt}.
}
\begin{enumerate}
\item {  the arrival of warm ocean water to Peru and Ecuador near Christmastime }
\item {  high atmospheric pressure over the eastern Pacific Ocean resulting in trade winds that blow toward the equator from the north and south }
\item {  a southern jet stream branch that flows eastward bringing higher rainfall to the southeastern United States and helping break apart Atlantic and Caribbean storms resulting in fewer hurricanes }
\item {  heavy rainstorms in the western United States }
\item {  all of these }
\end{enumerate}
\end{samepage}
}
\item {
\setlength{\itemsep}{0cm}
\setlength{\parskip}{.2cm}
\begin{samepage}
\textbf{
In the open ocean, tsunami can travel \makebox[1cm]{\Rivpt\hrulefill\Rivpt} miles per hour with periods up to \makebox[1cm]{\Rivpt\hrulefill\Rivpt} minutes.
}
\begin{enumerate}
\item { 	50; 20 }
\item { 	50; 60 }
\item { 	485; 20 }
\item { 	485; 60 }
\item { 	670; 60 		 }
\end{enumerate}
\end{samepage}
}
\item {
\setlength{\itemsep}{0cm}
\setlength{\parskip}{.2cm}
\begin{samepage}
\textbf{
The Pacific Ocean has an average depth of about 5,500 meters, which yields a theoretical deep-ocean tsunami velocity of about \makebox[1cm]{\Rivpt\hrulefill\Rivpt} meters per second.
}
\begin{enumerate}
\item { 	2 }
\item { 	23 }
\item { 	230 }
\item { 	2300 }
\item { 	23,000 		 }
\end{enumerate}
\end{samepage}
}
\item {
\setlength{\itemsep}{0cm}
\setlength{\parskip}{.2cm}
\begin{samepage}
\textbf{
Powerful tsunami are most frequently produced by \makebox[1cm]{\Rivpt\hrulefill\Rivpt}.
}
\begin{enumerate}
\item { 	volcanoes }
\item { 	underwater landslides }
\item { 	Earthquakes }
\item { 	impacts of asteroids }
\item { 	storms 		 }
\end{enumerate}
\end{samepage}
}
\item {
\setlength{\itemsep}{0cm}
\setlength{\parskip}{.2cm}
\begin{samepage}
\textbf{
Greenhouse gases include \makebox[1cm]{\Rivpt\hrulefill\Rivpt}.
}
\begin{enumerate}
\item {  CO\ensuremath{_2} }
\item {  H\ensuremath{_2}O vapor }
\item {  methane (CH4) }
\item {  chlorofluorocarbons }
\item {  all of these }
\end{enumerate}
\end{samepage}
}
\item {
\setlength{\itemsep}{0cm}
\setlength{\parskip}{.2cm}
\begin{samepage}
\textbf{
In which of the following scenarios would tsunami tend to have the greatest destructive power?
}
\begin{enumerate}
\item { 	A section of a linear coastline where the bottom of the ocean dips gently near the coast. }
\item { 	A section of a linear coastline where the bottom of the ocean dips steeply near the coast. }
\item { 	A section of a coastline where there is a harbor and the bottom of the ocean bottom dips gently. }
\item { 	A section of a coastline where there is a harbor and the bottom of the ocean bottom dips steeply. }
\item { 	Tsunami would have the same destructive power in all of the conditions listed above. }
\end{enumerate}
\end{samepage}
}
\item {
\setlength{\itemsep}{0cm}
\setlength{\parskip}{.2cm}
\begin{samepage}
\textbf{
About 14,000 years ago, the Earth began to warm according to changes seen in annual ice layers in glaciers: the O18, CO\ensuremath{_2}, and methane content in ice \makebox[1cm]{\Rivpt\hrulefill\Rivpt}.
}
\begin{enumerate}
\item {  decreased }
\item {  increased }
\item {  stayed the same despite orbital changes that brought warming }
\end{enumerate}
\end{samepage}
}
\item {
\setlength{\itemsep}{0cm}
\setlength{\parskip}{.2cm}
\begin{samepage}
\textbf{
The greenhouse effect results in \makebox[1cm]{\Rivpt\hrulefill\Rivpt}.
}
\begin{enumerate}
\item {  global cooling }
\item {  global warming }
\item {  lots of flowers }
\item {  the growth of green algae on the shady side of houses worldwide }
\end{enumerate}
\end{samepage}
}
\item {
\setlength{\itemsep}{0cm}
\setlength{\parskip}{.2cm}
\begin{samepage}
\textbf{
Which of the following is least significant as a greenhouse gas on Earth?
}
\begin{enumerate}
\item {  ozone }
\item {  methane }
\item {  water vapor }
\item {  helium }
\item {  CO\ensuremath{_2} }
\end{enumerate}
\end{samepage}
}
\item {
\setlength{\itemsep}{0cm}
\setlength{\parskip}{.2cm}
\begin{samepage}
\textbf{
Tsunami are deadly natural hazards that commonly are generated by \makebox[1cm]{\Rivpt\hrulefill\Rivpt}.
}
\begin{enumerate}
\item { 	fault motion with vertical offset under the sea during which there is vertical offset }
\item { 	fault movements on land during which there is vertical offset }
\item { 	tides produced by gravitational attraction between the Earth and the Moon }
\item { 	hurricanes }
\item { 	fault movements on land in which there is horizontal offset only 		 }
\end{enumerate}
\end{samepage}
}
\item {
\setlength{\itemsep}{0cm}
\setlength{\parskip}{.2cm}
\begin{samepage}
\textbf{
Glass is opaque to \makebox[1cm]{\Rivpt\hrulefill\Rivpt} radiation.
}
\begin{enumerate}
\item {  long-wavelength }
\item {  short-wavelength }
\item {  both short- and long-wavelength  }
\item {  neither short- nor long-wavelength }
\item {  You can see through glass, so it is transparent to all radiation. }
\end{enumerate}
\end{samepage}
}
\item {
\setlength{\itemsep}{0cm}
\setlength{\parskip}{.2cm}
\begin{samepage}
\textbf{
The level of oxygen in the form of O\ensuremath{_2} in the Earth's atmosphere \makebox[1cm]{\Rivpt\hrulefill\Rivpt}.
}
\begin{enumerate}
\item {  has not changed substantially in the past 450 million years }
\item {  is drastically decreasing due to pollution }
\item {  is no different than it was 4.6 billion years ago }
\item {  is drastically increasing due to growth of new bacterial strains }
\item {  cannot change because Earth is a closed system.  }
\end{enumerate}
\end{samepage}
}
\item {
\setlength{\itemsep}{0cm}
\setlength{\parskip}{.2cm}
\begin{samepage}
\textbf{
The Little Ice Age that affected Europe from about \makebox[1cm]{\Rivpt\hrulefill\Rivpt} lowered average annual temperature by only about 1\ensuremath{^\circ}C but was enough to reduce crop yields, cause mountain glaciers to advance, and produce winters much more severe than in the twentieth century.
}
\begin{enumerate}
\item {  1450 to 1850 C.E. }
\item {  100 to 800 C.E. }
\item {  1700 to 1900 C.E. }
\item {  800 to 1450 C.E. }
\item {  1450 to 1850 B.C.E. }
\end{enumerate}
\end{samepage}
}
\item {
\setlength{\itemsep}{0cm}
\setlength{\parskip}{.2cm}
\begin{samepage}
\textbf{
The cycles of slow buildup and advance of glaciers followed by rapid shrinkage and retreat is caused by \makebox[1cm]{\Rivpt\hrulefill\Rivpt}.
}
\begin{enumerate}
\item {  Eccentricity of the Earth's orbit around the Sun  }
\item {  Tilt of the Earth's axis }
\item {  Precession of the equinoxes  }
\item {  all of these }
\item {  none of these }
\end{enumerate}
\end{samepage}
}
\item {
\setlength{\itemsep}{0cm}
\setlength{\parskip}{.2cm}
\begin{samepage}
\textbf{
The largest wave during a tsunami event is \makebox[1cm]{\Rivpt\hrulefill\Rivpt}.
}
\begin{enumerate}
\item { 	the first }
\item { 	the third }
\item { 	the fifth }
\item { 	unpredictable, it could be any of them }
\item { 	none, the are all about the same size 		 }
\end{enumerate}
\end{samepage}
}
\item {
\setlength{\itemsep}{0cm}
\setlength{\parskip}{.2cm}
\begin{samepage}
\textbf{
Episodes of climate change during the last 1000 years include all but which of the following?
}
\begin{enumerate}
\item {  the Maunder Minimum }
\item {  the Medieval Maximum }
\item {  the Little Ice Age }
\item {  the Renaissance Warming }
\item {  All of the above were episodes of short-term climate change. }
\end{enumerate}
\end{samepage}
}
\item {
\setlength{\itemsep}{0cm}
\setlength{\parskip}{.2cm}
\begin{samepage}
\textbf{
Which event produces the biggest tsunami?
}
\begin{enumerate}
\item { 	Earthquake }
\item { 	Underwater landslides }
\item { 	Hurricanes }
\item { 	Impacts of asteroids and comets }
\item { 	Volcanoes 		 }
\end{enumerate}
\end{samepage}
}
