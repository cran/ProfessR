\item {
\setlength{\itemsep}{0cm}
\setlength{\parskip}{.2cm}
\begin{samepage}
\textbf{
One of the greatest weather disasters in U.S. history occurred during the \makebox[1cm]{\Rivpt\hrulefill\Rivpt}, when several years of drought turned grain-growing areas in the center of the nation into the 'Dust Bowl'. 
}
\begin{enumerate}
\item {  1920s }
\item {  1930s }
\item {  1950s }
\item {  1970s }
\item {  1990s }
\end{enumerate}
\end{samepage}
}
\item {
\setlength{\itemsep}{0cm}
\setlength{\parskip}{.2cm}
\begin{samepage}
\textbf{
The greenhouse effect results in \makebox[1cm]{\Rivpt\hrulefill\Rivpt}.
}
\begin{enumerate}
\item {  global cooling }
\item {  global warming }
\item {  lots of flowers }
\item {  the growth of green algae on the shady side of houses worldwide }
\end{enumerate}
\end{samepage}
}
\item {
\setlength{\itemsep}{0cm}
\setlength{\parskip}{.2cm}
\begin{samepage}
\textbf{
In the open ocean, tsunami can travel \makebox[1cm]{\Rivpt\hrulefill\Rivpt} miles per hour with periods up to \makebox[1cm]{\Rivpt\hrulefill\Rivpt} minutes.
}
\begin{enumerate}
\item { 	50; 20 }
\item { 	50; 60 }
\item { 	485; 20 }
\item { 	485; 60 }
\item { 	670; 60 		 }
\end{enumerate}
\end{samepage}
}
\item {
\setlength{\itemsep}{0cm}
\setlength{\parskip}{.2cm}
\begin{samepage}
\textbf{
Principal factors that come into play when volcanism affects climate include all but which of the following?
}
\begin{enumerate}
\item {  the size and rate of eruptions }
\item {  the heights of eruption columns }
\item {  the types of gases and the atmospheric level at which they are placed }
\item {  the latitude of the eruptions }
\item {  All of these factors are considered. }
\end{enumerate}
\end{samepage}
}
\item {
\setlength{\itemsep}{0cm}
\setlength{\parskip}{.2cm}
\begin{samepage}
\textbf{
The fraction of solar energy reflected back to space due to Earth's cloudiness or snow and ice cover is known as \makebox[1cm]{\Rivpt\hrulefill\Rivpt}.
}
\begin{enumerate}
\item {  libido }
\item {  albedo }
\item {  segundo }
\item {  La Nina }
\end{enumerate}
\end{samepage}
}
\item {
\setlength{\itemsep}{0cm}
\setlength{\parskip}{.2cm}
\begin{samepage}
\textbf{
Likely changes in the 21st century due to global warming include all but which of the following?
}
\begin{enumerate}
\item {  30 meter rise in sea level }
\item {  melting and retreat of the Greenland and west Antarctic ice sheets }
\item {  melting of permafrost in Arctic regions }
\item {  northward advance of forests inside the Arctic circle }
\item {  more frequent severe drought in the central United States }
\end{enumerate}
\end{samepage}
}
\item {
\setlength{\itemsep}{0cm}
\setlength{\parskip}{.2cm}
\begin{samepage}
\textbf{
Tsunami are deadly natural hazards that commonly are generated by \makebox[1cm]{\Rivpt\hrulefill\Rivpt}.
}
\begin{enumerate}
\item { 	fault motion with vertical offset under the sea during which there is vertical offset }
\item { 	fault movements on land during which there is vertical offset }
\item { 	tides produced by gravitational attraction between the Earth and the Moon }
\item { 	hurricanes }
\item { 	fault movements on land in which there is horizontal offset only 		 }
\end{enumerate}
\end{samepage}
}
\item {
\setlength{\itemsep}{0cm}
\setlength{\parskip}{.2cm}
\begin{samepage}
\textbf{
The climatic cooling of the last 55.5 million years has been caused by \makebox[1cm]{\Rivpt\hrulefill\Rivpt}.
}
\begin{enumerate}
\item {  the ongoing breakup of Pangaea into separate continents   }
\item {  continental masses moving into polar latitudes }
\item {  snow and ice accumulating on polar landmasses, increasing albedo.  }
\item {  the uplifts of the Tibetan Plateau/Himalaya Mountains in Asia and the Colorado Plateau in the western United States deflecting west-to-east atmospheric circulation in the midlatitudes. }
\item {  all of these }
\end{enumerate}
\end{samepage}
}
\item {
\setlength{\itemsep}{0cm}
\setlength{\parskip}{.2cm}
\begin{samepage}
\textbf{
Over the last 7000 years there has been a \makebox[1cm]{\Rivpt\hrulefill\Rivpt}.
}
\begin{enumerate}
\item {  lowering of global average temperature totaling 2\ensuremath{^\circ}C }
\item {  raising of global average temperature totaling 2\ensuremath{^\circ}C }
\item {  constant global temperature, on average  }
\item {  None of these are correct. }
\end{enumerate}
\end{samepage}
}
\item {
\setlength{\itemsep}{0cm}
\setlength{\parskip}{.2cm}
\begin{samepage}
\textbf{
The largest wave during a tsunami event is \makebox[1cm]{\Rivpt\hrulefill\Rivpt}.
}
\begin{enumerate}
\item { 	the first }
\item { 	the third }
\item { 	the fifth }
\item { 	unpredictable, it could be any of them }
\item { 	none, the are all about the same size 		 }
\end{enumerate}
\end{samepage}
}
\item {
\setlength{\itemsep}{0cm}
\setlength{\parskip}{.2cm}
\begin{samepage}
\textbf{
An event similar to a tsunami that can occur in lakes due to avalanches, earthquakes, and other mechanisms is called a \makebox[1cm]{\Rivpt\hrulefill\Rivpt}.
}
\begin{enumerate}
\item { 	seiche }
\item { 	splash }
\item { 	slosh }
\item { 	tidal wave }
\item { 	flood }
\end{enumerate}
\end{samepage}
}
\item {
\setlength{\itemsep}{0cm}
\setlength{\parskip}{.2cm}
\begin{samepage}
\textbf{
Powerful tsunami are most frequently produced by \makebox[1cm]{\Rivpt\hrulefill\Rivpt}.
}
\begin{enumerate}
\item { 	volcanoes }
\item { 	underwater landslides }
\item { 	Earthquakes }
\item { 	impacts of asteroids }
\item { 	storms 		 }
\end{enumerate}
\end{samepage}
}
\item {
\setlength{\itemsep}{0cm}
\setlength{\parskip}{.2cm}
\begin{samepage}
\textbf{
In which of the following scenarios would tsunami tend to have the greatest destructive power?
}
\begin{enumerate}
\item { 	A section of a linear coastline where the bottom of the ocean dips gently near the coast. }
\item { 	A section of a linear coastline where the bottom of the ocean dips steeply near the coast. }
\item { 	A section of a coastline where there is a harbor and the bottom of the ocean bottom dips gently. }
\item { 	A section of a coastline where there is a harbor and the bottom of the ocean bottom dips steeply. }
\item { 	Tsunami would have the same destructive power in all of the conditions listed above. }
\end{enumerate}
\end{samepage}
}
\item {
\setlength{\itemsep}{0cm}
\setlength{\parskip}{.2cm}
\begin{samepage}
\textbf{
Episodes of climate change during the last 1000 years include all but which of the following?
}
\begin{enumerate}
\item {  the Maunder Minimum }
\item {  the Medieval Maximum }
\item {  the Little Ice Age }
\item {  the Renaissance Warming }
\item {  All of the above were episodes of short-term climate change. }
\end{enumerate}
\end{samepage}
}
\item {
\setlength{\itemsep}{0cm}
\setlength{\parskip}{.2cm}
\begin{samepage}
\textbf{
The Older Dryas and Younger Dryas stages represented \makebox[1cm]{\Rivpt\hrulefill\Rivpt}.
}
\begin{enumerate}
\item {  two cooling stages during warmup from the last major glacial advance }
\item {  two of the more abrupt warming stages during the last glaciation }
\item {  one warming stage and one cooling stage during the last glaciation }
\item {  periods of dry climate  }
\end{enumerate}
\end{samepage}
}
\item {
\setlength{\itemsep}{0cm}
\setlength{\parskip}{.2cm}
\begin{samepage}
\textbf{
A strong El Nino is characterized by \makebox[1cm]{\Rivpt\hrulefill\Rivpt}.
}
\begin{enumerate}
\item {  the arrival of warm ocean water to Peru and Ecuador near Christmastime }
\item {  high atmospheric pressure over the eastern Pacific Ocean resulting in trade winds that blow toward the equator from the north and south }
\item {  a southern jet stream branch that flows eastward bringing higher rainfall to the southeastern United States and helping break apart Atlantic and Caribbean storms resulting in fewer hurricanes }
\item {  heavy rainstorms in the western United States }
\item {  all of these }
\end{enumerate}
\end{samepage}
}
\item {
\setlength{\itemsep}{0cm}
\setlength{\parskip}{.2cm}
\begin{samepage}
\textbf{
The destructive powers of tsunami result mainly from their \makebox[1cm]{\Rivpt\hrulefill\Rivpt}.
}
\begin{enumerate}
\item { 	incredible height }
\item { 	unpredictablility }
\item { 	cold water }
\item { 	momentum and long wavelength }
\item { 	none of these 		 }
\end{enumerate}
\end{samepage}
}
\item {
\setlength{\itemsep}{0cm}
\setlength{\parskip}{.2cm}
\begin{samepage}
\textbf{
Surface temperatures on Venus are \makebox[1cm]{\Rivpt\hrulefill\Rivpt} those on the surface of Earth.
}
\begin{enumerate}
\item {  much lower than }
\item {  a little bit lower, on average, than }
\item {  about the same as }
\item {  a little bit higher, on average, than }
\item {  much higher than }
\end{enumerate}
\end{samepage}
}
\item {
\setlength{\itemsep}{0cm}
\setlength{\parskip}{.2cm}
\begin{samepage}
\textbf{
About 14,000 years ago, the Earth began to warm according to changes seen in annual ice layers in glaciers: the O18, CO\ensuremath{_2}, and methane content in ice \makebox[1cm]{\Rivpt\hrulefill\Rivpt}.
}
\begin{enumerate}
\item {  decreased }
\item {  increased }
\item {  stayed the same despite orbital changes that brought warming }
\end{enumerate}
\end{samepage}
}
\item {
\setlength{\itemsep}{0cm}
\setlength{\parskip}{.2cm}
\begin{samepage}
\textbf{
Which event produces the biggest tsunami?
}
\begin{enumerate}
\item { 	Earthquake }
\item { 	Underwater landslides }
\item { 	Hurricanes }
\item { 	Impacts of asteroids and comets }
\item { 	Volcanoes 		 }
\end{enumerate}
\end{samepage}
}
\item {
\setlength{\itemsep}{0cm}
\setlength{\parskip}{.2cm}
\begin{samepage}
\textbf{
The Pacific Ocean has an average depth of about 5,500 meters, which yields a theoretical deep-ocean tsunami velocity of about \makebox[1cm]{\Rivpt\hrulefill\Rivpt} meters per second.
}
\begin{enumerate}
\item { 	2 }
\item { 	23 }
\item { 	230 }
\item { 	2300 }
\item { 	23,000 		 }
\end{enumerate}
\end{samepage}
}
