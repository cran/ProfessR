\documentclass[10pt]{article}
\usepackage{amsmath} %Never write a paper without using amsmath for its many new commands
\usepackage{amssymb} %Some extra symbols
\usepackage{makeidx} %If you want to generate an index, automatically
\usepackage{graphicx} %If you want to include postscript graphics
%%%  \usepackage{mystyle} %Create your own file, mystyle.sty where you put all your own \newcommand statements
\usepackage{amscd}
\newcommand{\mydegree}{ \ensuremath{^\circ} }
\newcommand{\Rivpt}{\rule{.1pt}{1pt}}
\clubpenalty10000
\widowpenalty10000
\raggedbottom
\addtolength{\topskip}{0pt plus 10pt}
\let\oldsubsubsection=\subsubsection
\renewcommand{\subsubsection}{%
\filbreak
\oldsubsubsection
}
\setlength\topmargin{0in}
\setlength\headheight{0in}
\setlength\headsep{0in}
\setlength\textheight{9in}
\setlength\textwidth{7in}
\setlength\oddsidemargin{-.3in}
\setlength\evensidemargin{-.3in}
\setlength\parindent{0.25in}
\setlength\parskip{0.25in}
\begin{document}
\begin{enumerate}
\item {
\setlength{\itemsep}{0cm}
\setlength{\parskip}{.2cm}
\begin{samepage}
\textbf{
The destructive powers of tsunami result mainly from their \makebox[1cm]{\Rivpt\hrulefill\Rivpt}.
}
ANSWER: d)	momentum and long wavelength
\end{samepage}
}
\item {
\setlength{\itemsep}{0cm}
\setlength{\parskip}{.2cm}
\begin{samepage}
\textbf{
The largest wave during a tsunami event is \makebox[1cm]{\Rivpt\hrulefill\Rivpt}.
}
ANSWER: d)	unpredictable, it could be any of them
\end{samepage}
}
\item {
\setlength{\itemsep}{0cm}
\setlength{\parskip}{.2cm}
\begin{samepage}
\textbf{
The Pacific Ocean has an average depth of about 5,500 meters, which yields a theoretical deep-ocean tsunami velocity of about \makebox[1cm]{\Rivpt\hrulefill\Rivpt} meters per second.
}
ANSWER: c)	230
\end{samepage}
}
\item {
\setlength{\itemsep}{0cm}
\setlength{\parskip}{.2cm}
\begin{samepage}
\textbf{
Indications of global warming in the last half of the 20th century include all but which of the following?
}
ANSWER: d) people noticing how much warmer the weather is
\end{samepage}
}
\item {
\setlength{\itemsep}{0cm}
\setlength{\parskip}{.2cm}
\begin{samepage}
\textbf{
One of the greatest weather disasters in U.S. history occurred during the \makebox[1cm]{\Rivpt\hrulefill\Rivpt}, when several years of drought turned grain-growing areas in the center of the nation into the 'Dust Bowl'. 
}
ANSWER: b) 1930s
\end{samepage}
}
\item {
\setlength{\itemsep}{0cm}
\setlength{\parskip}{.2cm}
\begin{samepage}
\textbf{
Limestone is rock composed of \makebox[1cm]{\Rivpt\hrulefill\Rivpt}. 
}
ANSWER: c) CaCO\ensuremath{_3}
\end{samepage}
}
\item {
\setlength{\itemsep}{0cm}
\setlength{\parskip}{.2cm}
\begin{samepage}
\textbf{
The Older Dryas and Younger Dryas stages represented \makebox[1cm]{\Rivpt\hrulefill\Rivpt}.
}
ANSWER: a) two cooling stages during warmup from the last major glacial advance
\end{samepage}
}
\item {
\setlength{\itemsep}{0cm}
\setlength{\parskip}{.2cm}
\begin{samepage}
\textbf{
The greenhouse effect results in \makebox[1cm]{\Rivpt\hrulefill\Rivpt}.
}
ANSWER: b) global warming
\end{samepage}
}
\item {
\setlength{\itemsep}{0cm}
\setlength{\parskip}{.2cm}
\begin{samepage}
\textbf{
Principal factors that come into play when volcanism affects climate include all but which of the following?
}
ANSWER: e) All of these factors are considered.
\end{samepage}
}
\item {
\setlength{\itemsep}{0cm}
\setlength{\parskip}{.2cm}
\begin{samepage}
\textbf{
What was the world like during the Late Paleocene Torrid Age? 
}
ANSWER: a) Most of the world was wetter and warmer.
\end{samepage}
}
\item {
\setlength{\itemsep}{0cm}
\setlength{\parskip}{.2cm}
\begin{samepage}
\textbf{
In the open ocean, tsunami can travel \makebox[1cm]{\Rivpt\hrulefill\Rivpt} miles per hour with periods up to \makebox[1cm]{\Rivpt\hrulefill\Rivpt} minutes.
}
ANSWER: d)	485; 60
\end{samepage}
}
\item {
\setlength{\itemsep}{0cm}
\setlength{\parskip}{.2cm}
\begin{samepage}
\textbf{
The human drama of the 'Dust Bowl' was captured in many articles and books, including \makebox[1cm]{\Rivpt\hrulefill\Rivpt}.
}
ANSWER: d) \emph{The Grapes of Wrath} by John Steinbeck 
\end{samepage}
}
\item {
\setlength{\itemsep}{0cm}
\setlength{\parskip}{.2cm}
\begin{samepage}
\textbf{
The eruption of \makebox[1cm]{\Rivpt\hrulefill\Rivpt} has been called the greatest eruption in historic times, killing about 10,000 people outright by pyroclastic flows and another 107,000 indirectly through famine and disease. 
}
ANSWER: a) Tambora in 1815 C.E.
\end{samepage}
}
\item {
\setlength{\itemsep}{0cm}
\setlength{\parskip}{.2cm}
\begin{samepage}
\textbf{
An event similar to a tsunami that can occur in lakes due to avalanches, earthquakes, and other mechanisms is called a \makebox[1cm]{\Rivpt\hrulefill\Rivpt}.
}
ANSWER: a)	seiche
\end{samepage}
}
\item {
\setlength{\itemsep}{0cm}
\setlength{\parskip}{.2cm}
\begin{samepage}
\textbf{
Which event produces the biggest tsunami?
}
ANSWER: d)	Impacts of asteroids and comets
\end{samepage}
}
\item {
\setlength{\itemsep}{0cm}
\setlength{\parskip}{.2cm}
\begin{samepage}
\textbf{
Powerful tsunami are most frequently produced by \makebox[1cm]{\Rivpt\hrulefill\Rivpt}.
}
ANSWER: c)	Earthquakes
\end{samepage}
}
\item {
\setlength{\itemsep}{0cm}
\setlength{\parskip}{.2cm}
\begin{samepage}
\textbf{
In which of the following scenarios would tsunami tend to have the greatest destructive power?
}
ANSWER: c)	A section of a coastline where there is a harbor and the bottom of the ocean bottom dips gently.
\end{samepage}
}
\item {
\setlength{\itemsep}{0cm}
\setlength{\parskip}{.2cm}
\begin{samepage}
\textbf{
About 14,000 years ago, the Earth began to warm according to changes seen in annual ice layers in glaciers: the O18, CO\ensuremath{_2}, and methane content in ice \makebox[1cm]{\Rivpt\hrulefill\Rivpt}.
}
ANSWER: b) increased
\end{samepage}
}
\item {
\setlength{\itemsep}{0cm}
\setlength{\parskip}{.2cm}
\begin{samepage}
\textbf{
The fraction of solar energy reflected back to space due to Earth's cloudiness or snow and ice cover is known as \makebox[1cm]{\Rivpt\hrulefill\Rivpt}.
}
ANSWER: b) albedo
\end{samepage}
}
\item {
\setlength{\itemsep}{0cm}
\setlength{\parskip}{.2cm}
\begin{samepage}
\textbf{
Tsunami are deadly natural hazards that commonly are generated by \makebox[1cm]{\Rivpt\hrulefill\Rivpt}.
}
ANSWER: a)	fault motion with vertical offset under the sea during which there is vertical offset
\end{samepage}
}
\item {
\setlength{\itemsep}{0cm}
\setlength{\parskip}{.2cm}
\begin{samepage}
\textbf{
Over the last 7000 years there has been a \makebox[1cm]{\Rivpt\hrulefill\Rivpt}.
}
ANSWER: a) lowering of global average temperature totaling 2\ensuremath{^\circ}C
\end{samepage}
}
\end{enumerate}
\end{document}
